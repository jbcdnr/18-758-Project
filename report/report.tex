\documentclass{article}

\usepackage{enumerate}
\usepackage{amsmath, amssymb}
\usepackage{graphicx}
\usepackage[framed,numbered,autolinebreaks,useliterate]{mcode}

\newcommand{\file}[1]{\section{#1}\lstinputlisting{../#1}}

\begin{document}

\title{18-758 Project Report}
\author{Jean-Baptiste Cordonnier - Thomas Mullins}
\date{\today}
\maketitle


\section{Pulse}

  

\section{Timing sync}

We use a simplified version of the timing synchronization we have seen in class. We use a random sequence of symbols that is highly uncorrelated, such that we see a high peak during the correlation of the message and the timing synchronization (see plot). Then we get $\hat\tau$ and know where the signal starts exactly. 

% Todo correlation plot

\section{One-tap equalizer}

We are using a one-tap equalizer for each segment of the message. We send 5-length symbols pilot every 120 symbols of information. The one tap equalizer uses the following formula :

\[
  h_0 = \frac{\text{txPilot} \cdot \text{rxPilot}}{\text{txPilot}^2}
  \hspace{2cm}
  \text{eqMessage} = \frac{\text{message}}{h_0}
\]

The equalization is very efficient as we can see on the following plot. Even if the pilot sequence is very short, the $h_0$ is still precise and equalize well the received segment of the message. It corrects the phase drift and the modulus of the signal. 

% Todo equalizer plot

\section{Constellation}

We are using a 4-PSK constelation. 

\section{Channel Coding}


\section{Conclusion}


\clearpage
\appendix
\file{params.m}
\file{receiver.m}
\file{transmitter.m}
\file{doTimingSync.m}
\file{doSampling.m}
\file{applyPulse.m}
\file{channelEncode.m}
\file{channelDecode.m}
\file{plotSignal.m}
\file{DTFT.m}

\end{document}
